\documentclass[titlepage]{article}
\usepackage{array}
\usepackage{enumerate}
\usepackage{graphicx}
\usepackage{listings}
\usepackage{tabularx}
\usepackage{setspace}
\usepackage{natbib}

\begin{document}

\author{Santana Mach}
\title{COMP 7036 Research Proposal \\ Hands-On Versus Simulation Training}
\date{December 13, 2011}
\maketitle{}

\tableofcontents
\pagebreak
\onehalfspacing

\section{Abstract}
The debate between advocates of in-class education and advocates of remote education have
been conflicting for years now; much like the one between hands-on and simulation laboratories.
Each claim that their methodology provides better utility for both the student and the
educator (whether the individual instructor or the institute).  The Information Technology
field of work is one that can be debated whether the costs of a classroom environment
and the time spent outweigh the benefits from this traditional method.\\\\
This proposal seeks to study compare the two educational environments and provide a
definitive answer for students of the field.  This study would not only give students the
best chance to find a job or a career, but it would also provide better workers for the
industry.  The first part of this process will be to look up employment rates
from educational institutes.  The second will be to contact IT companies or businesses
for interviews.  This data will define how the current batch of IT professionals learned
there trade.

\section{Introduction}

There has always been an unsettling debate between the advocates of in-class education
and the advocates of remote education.  The in-class advocates claim that it is important
to learn and work in a social environment with instructor and peers in-person to guide
you.  This will also lead to improved teamwork ability and social communications.  On
the other side, the remote advocates claim that technology has advanced enough that an
in-class environment is not necessary any longer.  \\\\
This study will be looking into this matter within the Information Technology (IT) field
of work.  Since most or all the work is or can be done on a computer, people will be using
their own machines at home with Internet.  Any assignments and work can be done remotely
and delivered through email or virtual drop-in service.  A simulated environment can also be
set up remotely and allow students to test their applications.  With all these advancements
in technology and the speed at which we can communicate, advocates believe that it is the
future of education. \\\\
The question that this study will attempt to conclude is ``Which type of employee or student
education environment, in-class or remote, provides more success within the Information 
Technology industry?"  

\clearpage

\section{Problem and Setting}

\subsection{Problems}

\subsubsection{Main Problem}
Which type of employee or student education environment, in-class or remote, provides
more success within the Information Technology industry?

\subsubsection{Subproblem 1}
What is the success rate of finding and remaining in a job through in-class
schooling/training compared to the remote process?

\subsubsection{Subproblem 2}
What are the influences of schooling/training through in-class experience
compared to remote training?

\subsubsection{Subproblem 3}
Are there any differences in benefits or limitations for the methodologies between males
and females?

\subsection{Hypotheses}

Employees and students who learn from an in-class environment gain cooperative and
social experience that will help in the job search and stability.\\
\\
IT professionals who learn from a remote environment obtain better skills for working
on individual projects.

\subsection{Delimitations}
This research study will only provide statistics from IT-based jobs

\subsection{Definitions}
This section will define several terms that are used in the research proposal.  This will clarify
any misconceptions or confusions for any of the terms to be used.

\begin{tabularx}{\textwidth}{lX}
Benefits & Knowledge and skills gained through the training method.\\\\
Cooperative Experience & Ability to work projects in a team environment.\\\\
Environment & Atmosphere and culture of the training or workplace.\\\\
In-class Training & Learning and improving skills in a classroom environment.\\\\
Influence & Change the student's or employee's way of thinking.\\\\
IT & Information Technology \\\\
Limitations & Knowledge or skills that the training method fails to teach.\\\\
Remote Training & Learning and improving skills through online services\\\\
Social Experience & Ability to communicate with peers effectively.\\\\
Stability & Good job security in their current jobs and/or careers.\\\\
Success Rate & Percentage of IT professionals with successful careers.\\\\
\end{tabularx}

\subsection{Assumptions}

\subsection{Importance of Study}
This study is important because it will help define training programs in the future.
Choosing the proper training program will ensure that the employees or students emerging
will have necessary skills for the job.  It will aid companies and educational institutes
when making decisions on how to implement their training programs.

\clearpage

\section{Literature Review}

\clearpage

\section{Data}

\subsection{Data Required and Means to Obtain the Data}
The problem being researched requires qualitative data from participants of hands-on
and simulation training.  

\subsection{Research Methodology}

\subsection{Data Treatment Per Subproblem}

\subsubsection{Subproblem 1}

\subsubsection{Subproblem 2}

\clearpage

\section{Researcher Qualifications}
I am student in the Bachelor degree program at British Columbia Institute of Technology.
My field of expertise is under network administration with both theory and programming.
I am currently enrolled in a course for Applied Research Methods in Software Development
in which the research proposal initiated.  I have also been involved in a project for
marketing research in which our team performed quantitative research methods.

\clearpage

\section{Outline of Proposed Study}

\clearpage

\section{Appendices}

\clearpage

\section{References}

\bibliographystyle{te}

\bibliography{references}{}

\clearpage

\begin{lstlisting}

\end{lstlisting}

\begin{figure}[htb]                                                                       
  \begin{center}
    %\includegraphics[width=0.9\textwidth]{imgs/transmission.png}
  \end{center}
  \caption{Overall Transmission Diagram}
  \label{fig:transmission}
\end{figure}

\end{document}
