\documentclass[titlepage]{article}
\usepackage{array}
\usepackage{enumerate}
\usepackage{graphicx}
\usepackage{listings}
\usepackage{tabularx}

\begin{document}

\author{Santana Mach}
\title{COMP 7036 Research Proposal \\ Hands-On Versus Simulation Training}
\date{December 13, 2011}
\maketitle{}

\tableofcontents
\pagebreak
\linespread{1.5}

\section{Introduction}


\section{Problem and Setting}

\subsection{Problem and Subproblems}

\subsubsection{Problem}
Which type of employee or student educational process, in-class or remote,
provides more success within IT-based jobs and careers?

\subsubsection{Subproblem 1}
What is the success rate of finding and remaining in a job through in-class
schooling/training compared to the remote process?

\subsubsection{Subproblem 2}
What are the influences of schooling/training through in-class experience
compared to remote training?

\subsubsection{Subproblem 3}
Are there any differences in benefits or limitations for the methodologies between males
and females?

\subsection{Hypotheses}

Employees and students who learn from an in-class environment gain cooperative and
social experience that will help in the job search and stability.\\
\\
IT professionals who learn from a remote environment obtain better skills for working
on individual projects.

\subsection{Delimitations}
This research study will only provide statistics from IT-based jobs

\subsection{Definitions}
This section will define several terms that are used in the research proposal.  This will clarify
any misconceptions or confusions for any of the terms to be used.

\begin{tabularx}{\textwidth}{lX}
Benefits & \\\\
Cooperative Experience & \\\\
Environment & \\\\
In-class Training & \\\\
Influence & \\\\
IT & Information Technology \\\\
Remote Training & \\\\
Social Experience & \\\\
Stability & \\\\
Success Rate & \\\\

Mental Danger & Possibility of the training to cause mental disorders.\\\\
Influence & The way the training affects the mentality of the personnel compared to the other type of training.\\\\
Hands-on Training & Acquiring skills through real-life training exercises.\\
Simulation Training & Acquiring skills through a virtual medium the imitates real-life conditions.\\
\end{tabularx}

\subsection{Assumptions}

\subsection{Importance of Study}
This study is important because it will help define training programs in the future.
Choosing the proper training program will ensure that the employees or students emerging
will have necessary skills for the job.  It will aid companies and educational institutes
when making decisions on how to implement their training programs.

\clearpage

\section{Literature Review}

\clearpage

\section{Data}

\subsection{Data Required and Means to Obtain the Data}
The problem being researched requires qualitative data from participants of hands-on
and simulation training.  

\subsection{Research Methodology}

\subsection{Data Treatment Per Subproblem}

\subsubsection{Subproblem 1}

\subsubsection{Subproblem 2}

\clearpage

\section{Researcher Qualifications}
I am student in the Bachelor degree program at British Columbia Institute of Technology.
My field of expertise is under network administration with both theory and programming.
I am currently enrolled in a course for Applied Research Methods in Software Development
in which the research proposal initiated.  I have also been involved in a project for
marketing research in which our team performed quantitative research methods.

\clearpage

\section{Outline of Proposed Study}

\clearpage

\section{References}

\clearpage

\section{Appendices}

\clearpage

\begin{lstlisting}

\end{lstlisting}

\begin{figure}[htb]                                                                       
  \begin{center}
    %\includegraphics[width=0.9\textwidth]{imgs/transmission.png}
  \end{center}
  \caption{Overall Transmission Diagram}
  \label{fig:transmission}
\end{figure}

\end{document}
